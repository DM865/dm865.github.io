\documentclass[handout,usepdftitle=false,aspectratio=169,smaller,compress,sans,fleqn,xcolor=dvipsnames,fleqn,table]{beamer}

% for transparencies

% \class[handout]{beamer}
% for handouts

% \documentclass[smaller,sans,mathserif,trans,notes=show]{beamer}
% show navigation map on top right
%\documentclass[smaller,sans,fleqn,xcolor=dvipsnames,fleqn,table,mathserif]{beamer}
% DO NOT show navigation map on top right
%\documentclass[smaller,compress,sans,fleqn,xcolor=dvipsnames,fleqn,table,mathserif]{beamer}
% SHOW only sec title on top right
%\documentclass[smaller,handout,notes=show,onlysecheader,sans,fleqn,xcolor=dvipsnames,fleqn,table,mathserif]{beamer} 
%\documentclass[ignorenonframetext,smaller,subsecheader,sans,fleqn,xcolor=dvipsnames,fleqn,table,stillsansserifmath,stillsansseriftext,stillsansserifsmall,stillsansseriflarge]{beamer}
%onlysecheader

%\documentclass[ignorenonframetext,smaller,aspectratio=169,compress,sans,fleqn,xcolor=dvipsnames,fleqn,table,stillsansserifmath,stillsansseriftext,stillsansserifsmall,stillsansseriflarge]{beamer}

%\documentclass[ignorenonframetext,smaller,subsecheader,sans,fleqn,xcolor=dvipsnames,fleqn,table,stillsansserifmath,stillsansseriftext,stillsansserifsmall,stillsansseriflarge]{beamer}

%try also :
%\documentclass[11pt]{article}\usepackage{beamerarticle}
\usepackage[english]{babel}
\usepackage[utf8]{inputenc} 
\usepackage[T1]{fontenc}

\usetheme{Odense}

% files to be included: intro.tex | {commands.tex, beamermysplit.sty}; close.tex


\usepackage{intro}
\usepackage{definitions}
%\usepackage{stefano}
%\usepackage{lstampl}
%\lstset{language=ampl}
%\usepackage{arydshln,leftidx,mathtools}% http://ctan.org/pkg/{arydshln,leftidx,mathtools}
%\setlength{\dashlinegap}{2pt}
%\usepackage{etoolbox}
%\AtBeginEnvironment{array}{\setlength{\arraycolsep}{0ex}}
%\setlength{\arraycolsep}{0.5mm}
%\addtolength{\arraycolsep}{0mm}
%\setlength{\mathindent}{0pt}

\graphicspath{{Figures/}{Figures-EMAA/}}
\usepackage{fancyvrb}
\RecustomVerbatimEnvironment{Verbatim}{Verbatim}{xleftmargin=0mm}


\title{{\color{black}\normalfont\small DM865}\\
{\color{black}\normalfont\small Heuristics and Approximation Algorithms}\\[2em]
%
%{\small Lecture 1}\\ 
Course Organization}


\author{\alert{Lene Monrad Favrholdt} and Marco Chiarandini}
\date{}

\begin{document}

\frame[plain]{%%%%%%%%%%%%%%%%%%%%%%%%%%%%%%%%%%%%%%%%%%%%%%%%%%%%%%%%%%%%%%%%%%%%%
\titlepage


%\begin{flushright} [\emph{Slides by Stefano Gualandi, Politecnico di
%    Milano}]
%\end{flushright}
}
%
%
%\frame{%%%%%%%%%%%%%%%%%%%%%%%%%%%%%%%%%%%%%%%%%%%%%%%%%%%%%%%%%%%%%%%%%%%%%
%  \frametitle{Outline}
%  \tableofcontents
%}
%


\section{List of Contents}

\begin{frame}%%%%%%%%%%%%%%%%%%%%%%%%%%%%%%%%%%%%%%%%%%%%%%%%%%%%%%%%%%%%%%%%%%
  \frametitle{Course Organization}

This course is about Discrete Optimization via:
\medskip\begin{itemize}

\item Approximation Algorithms
\item Heuristics 
\end{itemize}


\pause
\begin{columns}[T,onlytextwidth]
  \column{0.5\textwidth}
\textbf{Problems:}


\begin{itemize}
  \itemsep=1ex
\item Set Cover
\item TSP
\item SAT
\item Knapsack
\item Scheduling
\item Bin Packing
\item Bonus: the project problem
  \end{itemize}


\column{0.5\textwidth}
\textbf{Techniques:} 


\begin{itemize}
  \itemsep=1ex
\item LP-rounding
\item Primal-Dual
\item Randomized LP rounding

\medskip
\item Construction Heuristics
\item Local Search
\item Metaheuristics
\item Implementation Framework
\item Efficiency issues
\item Experimental Analysis
  \end{itemize}

\end{columns}

\end{frame}





\frame{
\frametitle{Schedule}

\begin{itemize}
\itemsep=2ex


\item Class schedule:
  %\medskip
  \begin{itemize}
    %\itemsep=3ex

  \item See course web page: \url{dm865.github.io}
  \item \url{mitsdu.sdu.dk}
  \item Changes can occur (reload web page often)
  \end{itemize}

\bigskip
\medskip
\pause
\item Workload:
\begin{itemize}
\item Intro phase: 42 hours
\item Skills training phase: 42 hours
\end{itemize}



\end{itemize}

}





  \begin{frame}%%%%%%%%%%%%%%%%%%%%%%%%%%%%%%%%%%%%%%%%%%%%%%%%%%%%%%%%%%%%%%%%%%%%%
    \frametitle{Communication media}
    \medskip\begin{itemize}\itemsep=3ex
    
\item Public Web Page [WWW] $\Leftrightarrow$ BlackBoard \url{e-learn.sdu.dk} [BB]\\
 (links from and to each others)
\item \alert{Announcements} in [BB]
\item \alert{Course Documents} in [BB] (unless linked from [WWW]) 
%\item \alert{Discussion Board} (anonymous) in [BB]
\item Personal email \url{lenem@imada.sdu.dk}, \url{marco@imada.sdu.dk}
\item Office visits

\medskip

\item \alert{Mid term evaluation} in class
%\item You are welcome to visit me in my office in working hours (8-16).
\end{itemize}

\end{frame}


\begin{frame}%%%%%%%%%%%%%%%%%%%%%%%%%%%%%%%%%%%%%%%%%%%%%%%%%%%%%%%%%%%%%%%%%%%%%
  \frametitle{Literature}

\begin{itemize}
\itemsep=3ex
\item[WS] David P. Williamson and David B. Shmoys. Design of Approximation Algorithms. Cambridge University Press. 2010.

  \item[MAK] W. Michiels, E. Aarts and J. Korst. \alert{Theoretical Aspects of
    Local Search}. Springer Berlin Heidelberg, 2007
\end{itemize}

\bigskip

Other articles
\end{frame}



\frame{%%%%%%%%%%%%%%%%%%%%%%%%%%%%%%%%%%%%%%%%%%%%%%%%%%%%%%%%%%%%%%%%%%%%%
  \frametitle{Evaluation (1/3)}

  \medskip
  \begin{itemize}
    \itemsep=3ex

  \item Two practical assignments:\\
    \textbf{Part I:}\\
    Local search and experimental analysis for a routing problem \\
    \textbf{Part II:}\\
    Builds up on Part I:\\
    Metaheuristics for a routing problem

  \item Oral exam in June.\\ No preparation, External censor, 7-scale grade.

  \end{itemize}
}


\frame{%%%%%%%%%%%%%%%%%%%%%%%%%%%%%%%%%%%%%%%%%%%%%%%%%%%%%%%%%%%%%%%%%%%%%
  \frametitle{Evaluation (2/3)}
\begin{itemize}\itemsep=3ex

\item The final grade depends on the three parts.

\item The assignments will be graded by the teacher and receive a score.

\item The assignments will be available to the censor at the oral exam.
   
  \item Ten minutes of the oral exam on the assignments\\
    Ten minutes on the other part of the syllabus (approximation algorithms).

  \item The assignments must be carried out in pairs. The pairs must
    change between part 1 and 2.

  \item Communication between groups not allowed.

\end{itemize}


}


\begin{frame}%%%%%%%%%%%%%%%%%%%%%%%%%%%%%%%%%%%%%%%%%%%%%%%%%%%%%%%%%%%%%%%%%%%%%
  \frametitle{Evaluation (3/3)}
  
\begin{itemize}\itemsep=3ex
    \item The assignments are meant for learning. There will be
      discussion classes where we will help you with your questions.
    \item You should be able to improve during the time of the course.\\
Hence it matters most to us what you can do at the end.
\item If you had trouble with one assignments you should still be able to get 12.
\item If you have done well in the exercises but know nothing at the
  time of the exam you should still be able to fail.
\item But this does not mean that the assignments should be taken
  lightly, they do have an influence and they are perhaps the part where
  you will really learn to solve problems.
\end{itemize}
\end{frame}

\begin{frame}%%%%%%%%%%%%%%%%%%%%%%%%%%%%%%%%%%%%%%%%%%%%%%%%%%%%%%%%%%%%%%%%%%%%%
  \frametitle{Practical Assignments}

  \medskip\begin{itemize}\itemsep=3ex
    \item  Really practical! $\leadsto$ programming in Python 3.\\
 (Refresh it until you have time).

\item You will be provided with a starting framework but expect quite
  some work. The problem is quite challenging.


\item It would be nice if you could use R for the analysis of
  data.\\ (Some code examples might be provided but R will be useful in
  your carrer, learn it!)
 
\end{itemize}
\end{frame}



\begin{frame}%%%%%%%%%%%%%%%%%%%%%%%%%%%%%%%%%%%%%%%%%%%%%%%%%%%%%%%%%%%%%%%%%%
  \frametitle{Practical Assignments: Contents}



  \medskip\begin{itemize}\itemsep=3ex
\item Algorithm \alertb{design}
\item \alertb{Modeling}
\item \alertb{Implementation} (deliverable and checkable source code)
\item Written \alertb{description}
\item (Analytical) and experimental \alertb{analysis}
\item Performance counts!
\end{itemize}

\pause
\bigskip
Currently evaluating a web submission with automatic check, execution and comparison.

\bigskip
We hope you will have fun!
\end{frame}






\begin{frame}<0|handout:0>%%%%%%%%%%%%%%%%%%%%%%%%%%%%%%%%%%%%%%%%%%%%%%%%%%%%%%%%%%%%%%%%%%
  \frametitle{Practical Assignments: Learning Objectives}


\medskip
\begin{itemize}
  \itemsep=2ex
\item \alert{model} a problem similar in nature to the ones seen in the course
  within the framework of local search and metaheuristics
%\item \alert{argue} about the different modeling choices arising from the theory
%  behind the components of constraint programming, including global
%  constraints, propagators, search and branching schemes.
%\item \alert{develop} a solution prototype in a constraint programming system
\item \alert{design} specialized versions of general purpose heuristics:
  construction heuristics and local search
\item \alert{develop} a solution prototype in a local search framework
\item \alert{undertake an experimental analysis}, report the results and draw
  sound conclusions based on them
\item \alert{describe} the work done in an appropriate language including pseudocode
\end{itemize}


  \end{frame}






\begin{frame}%%%%%%%%%%%%%%%%%%%%%%%%%%%%%%%%%%%%%%%%%%%%%%%%%%%%%%%%%%%%%%%%%%%%%
  \frametitle{Active Participation}

  \medskip\begin{itemize}\itemsep=3ex
  \item We expect you to stay up-to-date with the course\\
The literature is not mandatory to read, you can get along well with
slides and lecture notes.
    \item Please ask questions!
    \item Experiment and explore
    \item Work with others
    \item Give us feedback.
    \end{itemize}
  \end{frame}








\end{document}




